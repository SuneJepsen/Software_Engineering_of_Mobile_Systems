In this chapter we will discuss and reflect our challenges regards to our work with this thesis objectives. 



%********************************************************************
\section{Theoretically}
%********************************************************************
The learning curve has been steep regards to learning the electronic voting protocol. The literature is on a very high mathematical academic level seen from the perspective of a software developer. We see this is as a challenge that a protocol is only described for such a narrow audience, especially  one that requires knowledge of such a specific field as cryptography. Most of the concepts used to describe the protocol assumes a certain background knowledge. When reading the article on protocol there is often reference to other literature that the protocol is build upon. This makes the article hard to read, as the reading flow is interrupted. \\

\noindent
With this thesis we have tried to break down the protocol such that it is understandable for a broader audience. We have constructed this thesis such that the reader gradually gains the required mathematical knowledge to understand the basic elements of the protocol. The description of the protocol has been divided into a basic and a detailed description. The basic description will give the reader basic knowledge about the protocol and should provide enough knowledge for a simple implementation of the protocol. Of cause this simple implementation will not fulfill all the security requirements for an electronic voting application.


\subsection{Knowledge accumulation}
Starting this thesis we had the subjects of multiparty computation and secure secret sharing in mind, but after guidance with our supervisor we agreed on working with the electronic voting protocol described in \cite{Schoenmakers1999}. In order to connect the theoretically part of the thesis with the practical part, we felt that additionally knowledge about electronic voting in generally were needed. After reading other articles on the subject we got a better overview on the concepts of electronic voting. It turns out that the security requirements for electronic voting is one of the key elements linking the theoretical part together with the practical part. This research helped us being able to include different issues, that we by our self would not consider immediately, such as our consideration regarding \textit{Eligibility} and our reflection on the \textit{Individual verifiability}. In general our research on the subject have given us a more critical approach to which security parameter we will have to take into consideration. For example we have given a short description of different known attacks on the discrete logarithm problem. Knowing these different attacks gives knowledge for how large $q$ must be, which is essential for our practical part.   


%********************************************************************
\section{Practical}
%********************************************************************
We have spend much time on understanding the protocol and the security requirements regarding electronic voting. This has been costly for the practical part of this thesis. We have not been able to develop a complete application, but rather a proof of concept, where we have tested the protocol and our architectural strategies. We would like to have spend more time on the practical part so that we could have tested our QAS such as performance and the availability. Since these two attribute is central for a software architecture on a large scaled application. \\

\noindent
Basically the entire protocol is about doing computations such as exponentiation, multiplying, addition,  modulo reduction, hash functions and generating random large primes etc. Doing it and doing it right is a challenge - all these computation together  makes the system complex and challenging to debug. We created several proof of concept with small numbers, which was manageable to follow. We haven't found a optimal way yet for debugging the computations with large numbers. \\

\noindent
As developers we are concerned with achieving maintainability and readability of the code, therefor code quality is an important aspect to take into account. The protocol works with variables like $g$, $G$, $Y$, $Y^*$, $DLEQ$, and $Proof_u$ etc. We ask our self what is the best naming conventions for these variables. Our naming conventions is based on trying to be consistent and give variables signing names.  


%********************************************************************
\subsection{Generating Prime}
%********************************************************************
Generating very large primes is a computationally hard task and thus time consuming. Given the fact
that the prime $q$ used in our implementation is public known, it is easy to think that one could simply
use one or a set of very large hardcoded primes. However, by performing a large precomputation for a given
prime, an adversary can quickly calculate arbitrary discrete logarithms in that group. And efficiently reducing the computation cost for all targets that uses this group \cite{Adrian:2015:IFS:2810103.2813707}. 


%********************************************************************
\subsection{Individual Verifiability} \label{sec:discussion_individual_verifiability}
%********************************************************************
Here we will elaborate on these two informal requirements from section \ref{sec:analyzing_electronic_voting_secure_requirements} regarding \textit{Individual Verifiability}. 

\begin{enumerate}
    \item All votes from the ballot casting is included in the final tally.
    \item Every votes from the ballot casting is calculated correctly.
\end{enumerate}

\noindent
As stated earlier our job is to convince the voter about the above statements. If this is possible, we mean that we have argumented for our statements and thereby the security requirement. To elaborate these requirements we will need some illustration of the ballots and the tallying phase. Figure \ref{fig:discussion:ballot_casting_and_tallying_phase} shows a simplified list of all valid ballots from the ballot casting phase. Figure \ref{fig:discussion:ballot_casting_and_tallying_phase} also shows a simplified list of the multiplied shares and the corresponding proofs by each tally.\\

\noindent
As for the first statement we refer that anyone can verify the correctness of the votes and the consistency of each share. By definition all is public and anyone are able to verify the validity of the outcome of the proofs under the election.\\

\noindent
So if the list is trusted and accepted by everyone then with high probability we can say it is correct. To ensure that all ballots are included in the final tally we have to be convinced, that each of the shares are included in the tallying phase. Each tally will compute $Y^*$, $S^*$ and a proof based on the shares belonging to them. The key is that anyone are able to confirm the $Y^*$ to ensure that all shares are included in the computation of $Y^*$. As for the second statement all can verify that the decryption of the shares are done correctly. All information needed in order for calculating the final tally is publicly available on the bulletin board and thus anyone are able to recalculate the final tally.      

\begin{figure}[H]
    \centering
    \captionsetup[subfigure]{labelformat=empty}
    \begin{subfigure}[b]{0.45\textwidth}
        \begin{table}[H]
            \centering
            \begin{tabular}{|l|l|}
                \hline
                \multicolumn{2}{|l|}{Ballot casting} \\ \hline
                Ballot   & Valid              \\ \hline
                1        & Yes                \\ \hline
                2        & Yes                \\ \hline
                3        & Yes                \\ \hline
                4        & Yes                \\ \hline
                5        & Yes                \\ \hline
                6        & Yes                \\ \hline
            \end{tabular}
        \end{table}
        \caption{Ballots with proofs validation}
    \end{subfigure}
    \qquad % <----------------- SPACE BETWEEN PICTURES
    \qquad % <----------------- SPACE BETWEEN PICTURES
    \begin{subfigure}[b]{0.40\textwidth}
        \begin{table}[H]
            \centering
            \begin{tabular}{|l|l|l|l|}
                \hline
                \multicolumn{4}{|l|}{Tallying phase}       \\ \hline
                Tally & $Y^*$ & $S^*$ & Valid                    \\ \hline
                1     & 10 & 3  & Yes                      \\ \hline
                2     & 8  & 9  & Yes                      \\ \hline
                3     & 5  & 8  & Yes                      \\ \hline
                4     & 9  & 4  & Yes                      \\ \hline
                5     & 9  & 3  & Yes                      \\ \hline
                6     & 2  & 8  & Yes                      \\ \hline
            \end{tabular}
        \end{table}
        \caption{Tallying phase}
    \end{subfigure}
    \caption{Tables showing the states of a Ballots and Tally shares after each phase in a election}
    \label{fig:discussion:ballot_casting_and_tallying_phase}
\end{figure}



\section{Lesson learned}
Looking back, we should have been better at narrowing the scope of the theoretical part of this thesis.  However this is a hard task with our  experience within the field of cryptography. With our programming experience and our knowledge in the field of software we can certainly say it has not been a trivial task to implement and at the same time getting the required knowledge of the different parts of the protocol. As a consequence we had to prioritize the project using the elements of "The Project Manager's Three-Legged Stool". Here we have three elements time, cost and quality which are parameters in a project. For this thesis time and cost are relatively fixed, which leads to one parameter to adjust namely quality. As stated earlier there is still work for us to do, on the practical part of our application. \\

\noindent
Our choice of development environment has been Visual Studio with ASP.NET, C\# and Javascript as main programming languages. Based on our knowledge that these technologies is widely used within our profession, the choice was clear for us. We have encountered challenges in our process to gather knowledge about certain technical aspects, revolving different cryptographic libraries in our development environment. The documentation for these libraries covering our development environments have been poor. A consequence of this have been that we have use more time then planned, on our application. We have read a lot of forums and documentation using other technologies on this subject. An recommendation will therefor be that one should be careful when choosing the tools to implement an application in this field. 








