\documentclass{article}
\usepackage{amssymb}
\usepackage{amsmath}
\usepackage{amsthm} 
\usepackage{url}
\usepackage{xspace}
\usepackage{graphicx}
\usepackage{float}
\usepackage[danish]{babel}
\usepackage[utf8]{inputenc}
\usepackage[T1]{fontenc}
\usepackage[vlined]{algorithm2e}
\usepackage{pgfplots}
\usepackage{booktabs,array,rotating}

\newcommand{\svec}{\ensuremath{\mathbf{s}}\xspace}

\newcommand{\splitcell}[1]{%
  \begin{tabular}{@{}l@{}}#1\end{tabular}%
}

\begin{document}

\noindent
Quatilty attribute\\
Quatilty attribute drivers\\
QAW og Quality attribute scenario : skalerbarhed og sikkerhed\\
Tactics\\
3+1 viewpoints\\
Distributed patterns\\
Microservices: distributed randomness\\
Sikkerhed på en browser\\
Toplogi: stjerner ingen snakker med hinanden\\
Beslutning omkring valg af kodeplatform... er det arktikturmæssig beslutning... Måske decisional krutchen ??\\\\

\noindent
klient (voter)\\
tallier\\
oberserver (master authority)\\
bullutin board (server)\\
\noindent
Initielt\\
En klient melder sig til bullutin board. Får tildelt public værdier. \\
En tallier melder sig til bullutin board. \\\\
\noindent
Ballotcasting -> Nu starter voting ->  $DLEQ$ og $PROOF_U$ -> indenfor deadline\\
Tallying -> Optællingsfasen for hver \\
Master authority -> Endelig summering -> brute force -> publish samlet stemme -> en vote som ikke er valideret tæller ikke med -> logning af beregning \\\\

\noindent
Issue tables\\
Diskussion/ta stilling \\
Remote randomness / webbrowser / .NET\\
https\\
man må ikke dobbelt vote\\
store tal\\
hardware krav


\begin{description}
  \item[paralist] provides compact lists and list versions
    that can be used within paragraphs, helps to customize
    labels and layout
  \item[enumitem] gives control over labels and lenghts in all
    kind of lists
  \item[mdwlist] is useful to customize description lists,
    it even allows multi-line labels. It features compact lists
    and the capability to suspend and resume.
  \item[desclist] offers more flexibility in definition list
  \item[multenum] produces vertical enumeration in
    multiple columns
\end{description}


\begin{math}
\frac{abc}{xyz}
 \end{math}    
 

Figur \ref{fig:DLEQ_1}

% Information vedr.  pgfplots // grafer: 
% https://www.maths.adelaide.edu.au/anthony.roberts/LaTeX/pgfplotBasics.pdf

\begin{tikzpicture}
\begin{axis}[
        xmin=0, xmax=4,
        ymin=0, ymax=4,
        axis lines=center,
        axis on top=true,
        domain=0:1,
        title style={at={(0.5,-0.15)},anchor=north},
        title={\textbf{Kaspers graf}},
    ]
    \addplot[
        scatter,
        only marks,
        point meta=explicit symbolic,
        scatter/classes={a={black}},
    ]
    table[meta=label] {
        x       y       label
        1       2       a
        2       4       a
        3       3       a
        4       4       a
    };
\end{axis}
\end{tikzpicture} 




\begin{sidewaystable}
\caption{The ATAM Attribute Utility Tree}

\medskip

\begin{tabular}{ @{} l l >{\raggedright\arraybackslash}p{9.5cm} @{} }
\toprule
\bfseries\splitcell{Quality \\ Attribute}
  & \bfseries\splitcell{Attribute \\ Refinement}
  & \bfseries Scenarios \\
\midrule
Reliability
  & Fatal error recovery
  & Mechanisms must be in place to ensure any error that occurs is handled without
    causing the application to crash.
\\\addlinespace
  & Reliable Simulation
  & Calculate physics simulation results in a reliable manner.
\\
\midrule
Scalability
  & Scale with world size and complexity
  & Physical world size and complexity will have an impact on graphics hardware performance. 
    The resource requirements must scale linearly with the complexity of the graphical assets 
    on screen.
\\\addlinespace
  & Scale with vehicle complexity
  & System performance will be affected by the increase in vehicle complexity. The resources 
    requirements must scale in proportion to the complexity size.
\\\addlinespace
  & Scale with multiplayer size
  & On the multiplayer server side, the amount of concurrent connections into the 
    server is going to impact performance. Servers must have a known budget and resources 
    requirements must be known for concurrent connection sizes.
\\
\midrule
Performance
  & Consistent performance
  & Maintain framerate as simulation complexity grows.
\\
\midrule
Maintainability
  & Core isolation
  & Update core physics engine without affecting other parts
\\\addlinespace
  & Pluggable rendering engine
  & Different rendering engines (Direct3D, OpenGL, Mantle etc.)\ should be used 
    depending on architecture and user requirements.
\\
\midrule
Flexibility
  & Simulation Flexibility
  & Simulate various types of vehicles such as cars, boats and aircraft with same
    simulation engine.
\\\addlinespace
  & Scriptable Interface to Simulation
  & Read vehicle configurations and simulation parameters in from human readable files.
\\
\midrule
Configurability
  & System Settings
  & Capacity for various translations (and possible future translations) must be
    available in the architecture.
\\\addlinespace
  & Scenario Settings
  & Adjustable scenario parameters such as weather conditions, time of day, terrain
    conditions and various triggerable events must be accessible from the interface.
\\
\midrule
Internationalisation
  & Local Translation
  & System settings such as audio, graphics and interface must be settable from the
    interface and changes must be persistent.
\\\addlinespace
  & Unit Translations
  & Units must be displayable in various formats such as imperial and ISO.
\\
\bottomrule
\end{tabular}
\end{sidewaystable}
 


\begin{thebibliography}{9}

\bibitem{levfou05}
  Eric Levieil and Pierre-Alain Fouque,
  \emph{An improved LPN algorithm},
  Paris Cedex 05, France, 2005.
  
\bibitem{BKW03}
  A. Blum, A. Kalai and H. Wasserman.
  \emph{Noise-tolerant Learning, the Parity Problem, and the statistical Query Problem},
  Journal of the ACM 50,4, July 2003, pp. 506-519.
  
\bibitem{BOTRVA}
  Sonia Bogos, Florian Tramér, and Serge Vaudenay
  \emph{On Solving LPN using BKW and Variants},
  {IACR} Cryptology ePrint Archive, 2015.  
  

\end{thebibliography}
 
\end{document}





    


