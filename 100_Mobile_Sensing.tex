% Admir

%--------------------------------------- Begin: Chapter intro ----------------------------------------------

This chapter explores the concepts of mobile sensing. Different phenomena such as participatory and opportunistic sensing is explained and related to the product, BikeBus. Types of application types and context awareness is also touched upon.

%--------------------------------------- Begin: Types of mobile sensing ----------------------------------------------
\section{Types of mobile sensing}  
% TODO: 
% Apply definitions
% Explain difference between participatory and opportunistic sensing
% Relate to articles.
% Relate to app (start geofence -> continuous sensing -> end geofence)
% Reflect and discuss. 
Mobile sensing is the act of using mobile devices and an cluster of embedded sensors to sense and collect data. The data can then be used for learning both physical and social phenomenon. The information can be used to inform, share and persuade humans to an improvement, either physically or otherwise \textcolor{red}{[REF]}. 


Sensing consists of two types, participatory sensing and opportunistic sensing. Participatory sensing involves the user. The user is throughout the sensing loop actively engaged in the sensing, with control over determining how, when, what and where to sample sensor data. Opportunistic sensing on the other hand runs as a continuous background activity where the user is passively participating in the sensing loop \textcolor{red}{[REF]}.

BikeBus makes use of opportunistic sensing which is for example observable with the location tracking\textcolor{red}{[REF to Location Tracking section]} and activity recognition\textcolor{red}{[REF to Activity Recognition section]} functionality. The sampling happens continuously in the background while the user is biking a route, and ends at the end of the route or at a greater diversions \textcolor{orange}{[Do we have this functionality?]} from the route. 

\begin{defi}[\textbf{Participatory Sensing}]
Participatory  Sensing  emphasizes  the  involvement  of  citizens  and  community  groups  in  the 
process  of  sensing  and  documenting  life  where  they  live,  work,  and  play. \cite{Goldman2009}. 
\end{defi}

\begin{defi}[\textbf{Opportunistic Sensing}]
Opportunistic sensing shifts the burden of supporting an application from the custodian to the sensing system, automatically determining when devices can be used to meet application requests. \cite{Lane:2008:USS:1411759.1411763} 
\end{defi}

%

% \begin{enumerate}
%     \item  Participatory sensing
    
%     \begin{enumerate}
%         \item  User actively engages in the data collection activity

%         \item  Manually determines how, when, what, and where to sample
   
%     \end{enumerate}
            
%     \item  Opportunistic sensing
    
%         \begin{enumerate}
%         \item   Data collection runs as a continuous background activity

%     \end{enumerate}
% \end{enumerate}
%--------------------------------------- End: Types of mobile sensing ----------------------------------------



%--------------------------------------- Begin: Application Types -------------------------------------------
\section{Application Types}
% TODO: 
% Apply definitions
% Explain difference between indivudual, group and community
% Relate to articles.
% Relate to app 
%   - App is individual (now)
% Reflect and discuss. 
%   - Group could be used for sensing (location)
%     precision/accurarcy (future)
%   - Community could be to provide route data for 
%     municipality (future)

\textcolor{red}{Definitions to be added!}

Sensing can be applied to different types of applications. These application sensing types consist of, individual activity sensing, group activity sensing and community activity sensing. Individual sensing is down on a personal level, where the application is designed for an individual alone and often revolves about providing the user with information through collecting and analysing data. Group...\textcolor{green}{TO BE CONTINUOUED!}  \textcolor{red}{[REF to A Survey of Mobile Phone Sensing]}
    % \begin{enumerate}
    %     \item  Individual activity sensing fitness applications, behavioural suggestions

    %     \item  Group activity sensing groups to sense common activities and help achieving group goals. Eg: assess neighbourhood safety, collective recycling efforts 
        
    %     \item Community sensing large scale sensing, where a large number of people have the same application installed. E.g., tracking spread of disease across a city, congestion in a city.
    % \end{enumerate}
    
    
%--------------------------------------- End: Application Types --------------------------------------------

    
%--------------------------------------- Begin: Context Awareness --------------------------------------------
\section{Context Awareness}
% TODO:
% Apply definitions
% Explain context awareness, classification and phone context
% Give example of how to combine multiple sensors
%   - ex: gyroscope and accelerometer
% Relate to articles
% Relate to app: how we use classification
% Reflect and discuss. Can context awareness be used in other ways in our app?

% Relate to activity recognition (uses accelerometer


\textcolor{red}{Definitions to be added!}






%--------------------------------------- End: Context Awareness --------------------------------------------
