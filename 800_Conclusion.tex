In this thesis three objectives were studied: The theory behind Shamir secret sharing and multiparty computation as well as the cryptographical concepts needed to understand it. The electronic voting protocol and the mathematical justification behind it. Design and implement an secure and scalable web based electronic voting application based on the protocol. We will in the following summarize our main achievements. \\

\noindent 
By combining the theoretical knowledge gained through session with our supervisor and through studying the field of cryptography, with applied practical experience gained by implementing the protocol in a proof-of-concept application. We have accomplish, most of the objectives in this thesis.

\begin{enumerate}
    \item We have gained comprehensive knowledge in regards to Sharmirs secret sharing and multiparty computation as well as the concepts surrounding them. Furthermore we feel that we have managed, through elaboration and the use of examples, to describe these concepts in such a way that peers with a similar backgrounds as us, as a software developers, can learn the concepts with a lesser steep learning curve. 
    
    \item We have taken the description of the PVSS protocol and the electronic voting protocol described in \cite{Schoenmakers1999} merging them together, for then to divide the description into three different parts each elaborating the protocol further. We see this structure as an optimal way for learning the protocol while trying the concepts in practice, as it allows for iterative implementing the protocol. 
\end{enumerate}    


\noindent  
Gaining knowledge about the theoretical concepts behind the protocol, have helped us understand how to implement the protocol. Even though we have practical programming experience, we none-the-less faced several situations where the theory helped us clarify how to solve the situation.    
    
\begin{enumerate}   
    \item While implementing the electronic voting application we had the chance to combine the theoretical knowledge with our programming experience. This has been a huge advantage in our learning process. One thing is to understand the mathematics behind the protocol. Another thing is knowing what is needed in order to implement the protocol. 
    We have found it beneficial to gain comprehensive knowledge of the mathematics and cryptography concepts behind, not only the protocol but also electronic voting in general. This unfortunately have been more time consuming then expected, which in the end had the consequence that we have not yet reach our objective with our application. Nevertheless  
    we got insight into designing and implementing a cryptographic protocol.
    
    \item Our design of the electronic voting application have been heavily influenced by the security requirements for electronic voting. We see these requirements as an essential part of developing an electronic voting application. While most of the requirements is taken into account by the protocol it self, there where some requirements that we had to incorporate into the application our-self. Even though our web based electronic voting application is not finished, we have prepared an architecture which is build upon known design principles from \cite{Bass} and \cite{Baerbak10} such as Quality attributes (QA), Quality attribute scenarios, tactics and design patterns. Furthermore we would like to highlight these QA, \textit{Interoperability}, \textit{Modifiability}, \textit{Security} and \textit{Testability} which the architecture is build upon.        
\end{enumerate}