% Admir
\subsection{Background}
% \textcolor{red}{Motivation and Context: Describe the project’s motivation and the respective context right here. In this part, also include references to literature that provides extra arguments.}


Multiple studies have documented that psychical activity has a disease preventive effect \cite{jensen2007cykling}. Other studies shows that psychical activity has a positive impact on people's learning ability regardless of their age \cite{fysiskAktivitet}. Therefor people need to exercise more, and preferably start as young as possible.

In Denmark we have really strong biking traditions. We need to hold onto this biking culture and pass it on to our children so that they can get the same joy of biking and its benefits. 

To improve the biking environment the government has initiated a "green" strategy which focuses on biking \cite{DanmarkOpPaaCyklen}. The bike is a cheap, healthy and a pollution-free vehicle. Other benefits could be fewer accidents and less noise. There is a need for initiatives and innovation that promote the joy of continuous use of biking.

Odense is the city of biking. In 2015 Odense received a price for best biking city \cite{OdenseCycleCity2015}. Biking is highly prioritized in the city infrastructure. Different bike paths and bridges have been constructed over the past years. In different places in the city there are sensor stations where specialized software can register cyclists and collect data. 

We need to make it more attractive and easier to choose the bike for work or school, rather than picking the car. We can do this by creating better bike paths, fewer stops, secure biking parking spaces and new bike facilities. Another strategy is to make it fun to use the bike every morning by either competing or collaborating with friends.

One initiative is the concept CykelScore, which the main idea is to encourage mainly children to bike in the city by using gamification. CykelScore has put up multiple bike stations around the city of Odense municipality containing a chip to register whenever a cyclist passes by. Every time a cyclist passes a CykelScore station they get a point, and they can collect all their points in a mobile application and compete with their friends.

This leads to a core observation namely how we as software engineering's can get the most from software which encourage and motivate the use of biking and thereby support the many reasons for biking and doing psychical activity at the same time. 

Our project focuses on how friends can collaborate and help each other use the bike more often by creating "Bike Buses", and thus motivating both children and adults to bike together.